\chapter{Symulacja zjawiska - implementacja i szczegóły techniczne}

\section{Wybór języka programowania}

Językiem który postanowiliśmy wybrać do implementacji symulacji przedstawionego powyżej problemu wybrano język Python w wersji \textbf{3.7.x}.

Ze względu na jego prostotę, wydajność i obszerny wybór bibliotek zewnętrznych wybraliśmy zamiast innych, również bardzo wydajnych ale bardziej skomplikowanych języków takich jak C++ czy Java.

% \clearpage

\section{Narzędzia wykorzystywane w trakcie tworzenia projektu}
\begingroup
\subsection{Środowisko programowania}
\begin{wrapfigure}{L}{0.55\textwidth}

	\centering
	\includegraphics[width=0.25\textwidth]{PyCharm_Logo.png}
	\caption{\label{fig:frog2}PyCharm Community}

\end{wrapfigure}

Wykorzystano \textbf{PyCharm Community 2019.3}, rozbudowane, potężne i wyposażone w dużą ilość dodatkowych narzędzi środowisko stworzone do pracy z językiem Python. Dostępne jest w dwóch wersjach, Professional oraz Community będąca wersją open-source przez co idealnie nadała się do wykorzystania przy naszym projekcie.

\endgroup
\clearpage

\subsection{Wybrane biblioteki zewnętrzne języka Python}
Jak wspomniano wcześniej, Python jest bardzo popularnym językiem przez co posiada wiele bibliotek zewnętrznych.

Do implementacji symulacji problemu wykorzystano następujące biblioteki zewnętrzne:

\begin{enumerate}
	\item \textbf{PyQt5} - oferująca zestaw narzędzi do budowy graficznego interfejsu użytkownika
	\item \textbf{pyowm} - oferująca zestaw narzędzi do eksploatacji API Open Weather Map
	\item \textbf{laspy} - zestaw funkcji umożliwiająca wykonywanie wielu operacji na plikach .las
	\item \textbf{paramiko} - moduł umożliwiający komunikację SSH z poziomu skryptu w Pythonie
	\item \textbf{pyvista} - biblioteka zawierająca implementację algorytmu triangulacji Delaunaya
\end{enumerate}

\subsection{Usługi chmurowe Google Cloud Platform i dane pogodowe}
\begingroup
\begin{wrapfigure}{L}{0.45\textwidth}
	
	\centering
	\includegraphics[width=0.25\textwidth]{gcp_logo.png}
	
\end{wrapfigure}
Kluczowym, z punktu widzenia projektu, okazało się znalezienie rozwiązania problemu ciągłej aktualizacji danych pogodowych. Z powodu iż dane muszą być nieustannie aktualizowane co 8 godzin, jedna z maszyn roboczych autorów musiałaby pracować bez przerwy. Nie jest to w żadnym stopniu rozwiązanie optymalne. Zdecydowano się na skorzystanie z darmowego okresu próbnego oferowanego na platformie \textbf{Google Cloud Platform}. Oferta Google zawarła w sobie możliwość stworzenia instancji maszyny wirtualnej operującej na systemie \textbf{Debian 10 Buster} co otworzyło drogę do rozwiązania problemu ciągłej aktualizacji danych pogodowych.
\clearpage

Dane pogodowe pobierane są co 8 godzin, o godzinach 00:00, 08:00 i 16:00 przy pomocy skryptu \textbf{weather\_conditions.py} który korzysta z biblioteki \textbf{pyowm}
\begin{lstlisting}[language=Python,caption=pseudokod realizujący pobieranie danych pogodowych]
 weather_conditions.py
 
 import pyowm

 # UTWÓRZ ODPOWIEDNI FOLDER JEŻELI NIE ISTNIEJE
 if dir_not_exists:
 	create_dir
 	
 # WSPÓŁRZĘDNE ODPOWIEDNIEGO OBSZARU
 coords = extract_coords(map_name)
 
 # POBRANIE DANYCH POGODOWYCH
 owm = pyowm.OWM("API_KEY")
 observation = owm.weather_around_coords(N_lat, W_long))
 
 # USUWANIE NAJSTARSZEGO POMIARU I ZASTĄPIENIE GO NOWYM
 if measurement_count == 6:
 	delete(oldest_measurement)
 	
 # STWÓRZ PLIK ZAWIERAJĄCY AKTUALNE DANE POGODOWE
 data_file.create(path_to_file)
 data_file.write(observation.weather_data)
\end{lstlisting}

\subsubsection{Harmonogram pobrania danych}
Za odpowiedni czas pobrania danych odpowiada narzędzie uniksopodobnych systemów operacyjnych \textbf{cron}. Jest to realizowane za pomocą odpowietnich wpisów w pliku \textbf{crontab}
\begin{verbatim}
	0 6 * * * /usr/bin/python /path_to_script/weather_conditions.py

	0 14 * * * /usr/bin/python /path_to_script/weather_conditions.py
	
	0 22 * * * /usr/bin/python /path_to_script/weather_conditions.py	
\end{verbatim}


