\chapter{Wprowadzenie}
\label{cha:wprowadzenie}

\section{Cel}
Celem niniejszej pracy jest stworzenie symulacji, która dostarcza w czasie rzeczywistym informacji dotyczących możliwego zagrożenia lawinowego na terenie Tatr w Polsce. 

%---------------------------------------------------------------------------

\section{Opis problemu}

Lawiny śnieżne są powszechnie występującym zagrożeniem na całym świecie i stanowią niebezpieczeństwo zarówno dla ludzi, jak i~biznesu oraz infrastruktury. Instytucje w różnych państwach alarmują o zagrożeniu lawinowym w podobny sposób, choć można zauważyć, że im większe zagrożenie stanowi zjawisko samoistnego ruchu śniegu, tym bardziej szczegółowe są prognozy. 

Głównym problemem przy określaniu ryzyka lawinowego jest złożoność tego zjawiska. Zależy ono zarówno od czynników stałych, do których zalicza się np. ukształtowanie terenu (1. typ uwarunkowań) oraz zmiennych, dotyczących stanu śniegu (2. typ uwarunkowań) oraz warunków meteorologicznych (3. typ uwarunkowań) (Woszczek 2016). Polskie instytucje wydają regularnie tzw. komunikaty lawinowe, są one jednak jedynie ogólnym zapisem zagrożenia - ratownicy zastrzegają, że informacje zawarte w komunikacie stanowią tylko podstawę do samodzielnej oceny. 

Przedmiotem poniższej pracy jest więc próba zastosowania technologii, by zapewnić system wspomagający decyzje ekspertów w celu tworzenia jeszcze bardziej precyzyjnych ostrzeżeń. Symulacja sama w sobie nie stanowi profesjonalnego narzędzia, ale zawiera koncepcje i rozwiązania, które można zastosować przy tworzeniu niezawodnego i złożonego narzędzia dla państwowych instytucji. 

\section{Możliwe rozwiązanie}
Na świecie modele przewidujące zagrożenie są tworzone w oparciu o metody statystyczne oraz metody uczenia maszynowego takie jak analiza najbliższego sąsiedztwa (ang. nearest neighbor analysis), analiza skupień (ang. cluster analysis), czy drzewa klasyfikacyjne (ang. classification trees) (Joshi, Kumar, Srivastava, Sachdeva, Ganju 2018). Są to jednak rozwiązania stworzone na podstawie wieloletnich pomiarów (również dotyczących warunków pokrywy śnieżnej), do których nie uzyskano dostępu. Zdecydowano się więc oprzeć na wnioskach autorów publikacji, charakteryzujących najważniejsze czynniki stwarzające ryzyko. 
 
W celu jak najdokładniejszego określania ryzyka wykorzystano bardzo precyzyjne informacje dotyczące ukształtowania terenu oraz dane pogodowe (dane dotyczące śniegu nie są ogólnodostępne i ich zmierzenie wymaga specjalistycznej wiedzy oraz narzędzi). 

Korzystając z danych topograficznych uproszczono ukształtowanie powierzchni Tatr (każdy z obszarów  o powierzchni ponad 2 km$^2$ reprezentowany jest przy pomocy około 110 punktów) i obliczono odpowiednie cechy. Następnie w połączeniu z cechami dotyczącymi warunków atmosferycznych możliwe stało się określenie ryzyka dla każdego takiego obszaru przy pomocy stworzonego wcześniej drzewa decyzyjnego. Dzięki takiemu podejściu, cały obszar Tatr Polskich podlega obserwacji, a w razie wystąpienia sprzyjających warunków (wykorzystano 2 z 3 istniejących uwarunkowań) w sposób zautomatyzowany wydaje się odpowiednie ostrzeżenia.

%---------------------------------------------------------------------------


W rodziale~\ref{cha:pierwszyDokument} przedstawiono podstawowe informacje dotyczące struktury dokumentów w \LaTeX u. Alvis~\cite{Alvis2011} jest językiem 


















